%
% Chapter 7
%
\chapter{Development of a Formal Testing Procedure}
\section{Initial considerations}
The new procedure was desgined as a direct response to the previous one. Based on complaints and difficulties from all who had to use the setup, we compiled a list of the biggest concerns and focused the procedure design around them. These included:
\begin{itemize}
	\item Manually changing input/output channels was slow and introduced too many steps
	\item Input \& output were connected through a flimsy probe clip
	\item No way to separate amp power rails
	\item No indication if the amps were receiving sufficient power
	\item No data collection for comparison between cards
	\item No standards or documentation for gain/offset potentiometer tweaks
\end{itemize}
\section{Design Choices}
The first readily apparent issue lay in the fact that there was a great deal of manual setup being performed by testers which was not ideal for long-term or large-scale testing. Amp cards were delivered in relatively small batches and had to be reworked and tested as they came. Having to search for a procedure that was hard to understand was not a good solution. I opted to use Python drivers communicating over the two serial specifications (USB serial and VX11 over Ethernet) to automate the input generation and output measurement. By having direct software access to the wave being generated and the data collection, users were in control of a closed loop from beginning to end. The remainder of the issues remained in the lack of hardware support for testing. Input and output locations were unclear and pin placement was only known to those who created the boards, unless testers wanted to open the PCB files and check themselves. By designing a pair of new PCBs to solve this issue, the test station had a clear set of instructions and devices to measure and capture data as was needed.

\section{Software Control}
Controlling the instruments available in our lab took a bit of research. While other group members had used some of the automation tools before, this was one of the first fully scripted hardware test setups in the lab. The communication protocols and code libraries required are explored below.
\subsection{pySerial Overview}
PySerial is a simple Python library to build and push serial commands over a variety of protocols. This library allows programmers to script serial access and build their own commands to be used for any device they require. For this application, I needed to build commands for a function generator. Other members of the group had previously automated this device and were able to provide documentation on the serial commands necessary. The benefit of a simple serial protocol was that it required no proprietary software and could be modified or extended for other devices. Initiating communication with a serial device is simple with pySerial as the programmer can simply can all serial ports and view device properties. In this specific application the serial device was the only one connected to the host computer, which allowed the device detection to be quite straightforward:\par
\begin{lstlisting}
	ports = list(serial.tools.list_ports.comports())
	for port, desc, hwid in sorted(ports):
		print("{}: {} [{}]".format(port, desc, hwid))
	for p in ports:
		print("USB Serial: {}".format(p.description))
		if "USB-SERIAL" in p.description:
				print("Found the FeelTech AWG")
				awg = feeltech_awg.FeelTechAWG(p[0])
	assert awg is not None, "No AWG device found"
\end{lstlisting}
To understand the process of communicating with this device I traced example commands to their lowest levels. To command the device to do anything, requests first had to be packaged into a recognizable prefix. To generate a prefix, parameters must be packaged in a particular order: \par
\begin{lstlisting}
	assert rw in range(2), "R/W value for prefix must be 0 (read) or 1 (write)"
	prefix_rw = "W" if rw == 1 else "R"
	assert channel in (1, 2), "Invalid channel, must be set to 1 (main) or 2 (auxilary)."
	prefix_channel = "M" if channel == 1 else "F"
	valid_modifiers = ("W", "F", "A", "O", "D", "P", "N")
	assert modifier in valid_modifiers, "Invalid cmd modifier (%s), must be in %s"%(modifier, valid_modifiers)
	return prefix_rw + prefix_channel + modifier
\end{lstlisting}
This command prefix is then concatenated with a desired parameter based on the function being controlled. Turning on an output channel, for example, takes in an output flag and attaches it to the command prefix:
\begin{lstlisting}
	"""
	output:  int -> 0 to turn channel off, 1 to turn channel on
	"""
	cmd = _generate_cmd_prefix(1, channel, "N") + str(output)
\end{lstlisting}
The command is now ready to be sent to the device. Using pySerial functions for read/write operations, the code is simple:
\begin{lstlisting}
	self.device.flushInput()
	self.device.write((cmd_str + "\n").encode())
	self.device.flush()
\end{lstlisting}
The straightforward nature of serial communications made this easy to debug and test. The function generator only needed to perform a few tasks (turn channels on/off, set waveform parameters) the scope of my research into its functionality was limited. I believe, however, that this establishes a good base for future testing development. Many other instruments use specifications such as VISA which may be a better choice in the future.
\FloatBarrier
\subsection{VXI11 Overview}
The VXI11 Communication Protocol is an architecture based on the VMEbus architecture meant for electronic communication. VXI11 in particular was built as an instrument communication spec \cite{vxi}. VXI11 requires three channels and operates over Ethernet.\par
VXI11 Python libraries are available and make VXI communication simple. Similarly to serial, commands must be constructed by the programmer and packaged into strings sent over the VXI bus. For example, writing to the device makes a device-level system call to write the input data to a predetermined location. From the programmer's point of view, one only has to use the high-level functions to facilitate communication. There are, for example, functions for read(), write(), and ask(), which performs a write and then read. These functions are grouped and abstracted for use with this specific instrument. If, for example, the programmer wants to ask the device what a selected channel's probe ratio is, the code is as follows:
\begin{lstlisting}
	def get_probe_ratio(self, channel):
		"""
		Returns the probe ratio for a specific channel
		"""
		channel = self._interpret_channel(channel)
		return float(self.query(':{0}:PROBe?'.format(channel)))
\end{lstlisting}
The string being pushed in the query function is a device-specific string that is outlined by the manufacturer \cite{rigol}. Using the listed commands and an ethernet connection, the VXI11 bus will automatically detect when a valid device is connected to the local network through Ethernet. The IP address of the device is specified upon initialization and needs to be checked before running tests.

\section{PCB Design Considerations}
While only a limited amount of hardware was required to drive the setup, design considerations had to be made to ensure there were no issues with single integrity or connector placement. The main goals of the test PCBs were to:
\begin{itemize}
	\item Provide users with a clear start and end point for signals
	\item Clearly label all relevant components
	\item Minimize manual steps to reduce error
	\item Provide instructions directly on the boards as well as official documentation
\end{itemize}
To accomplish this a thorough re-examination of the available equipment and possible solutions was required. To make the most of the previously mentioned measurement tools, simple shielded connectors were chosen as the entry and exit points for all signals. While these were not strictly necessary, standardizing connectors and overcompensating for signal integrity create stable foundations for future testing procedures. \par
Visual aids such as text and indicator displays were also necessary. It is impractical for a user to physically ensure power is being delivered when an active component could always sense voltage on its line. With new instructions, procedure and documentation, the visual aids were added purely as concessions for accessibility.\par
The final factor in the PCB design was proper electrical structure. The amplifier architecture provided separate ground and power lines for each amplifier component which housed four amplifiers each. In the system architecture, these ground planes and power lines were provided by the interface board (signaloutput) and DAC cards (input). Each of these ends unified the grounds. To mimic this architecture ground planes were added on both sides of the test setup to eliminate any risk of cross-talk or inaccurate results. The new testing hardware platform was built from the ground up as a representation of what the amps would be expected to do in the system.
\section{Emulating DAC input}
DAC output signals had previously been investigated with the purposes of limiting variations between analog response of individual channels. A long bout of experimentation yielded a list of voltage outputs from each DAC at different colors. As mentioned before, each ``color" band represents a range of values to represent desired brighness or intensity. These voltage levels provided a starting point for the voltage sweeps necessary to fully validate the amp cards. Results are recorded below:

\begin{table}[h!]
	\centering
	\caption{DACIN Value at Target Pixel Voltage}
	\begin{tabular}{ |c|c| }
		\hline
		\textbf{Voltage Level (V)} & \textbf{DACIN Value (Color Counts)} \\
		\hline
		5                          & 32767                               \\ \hline
		4                          & 26214                               \\ \hline
		3                          & 19660                               \\ \hline
		2                          & 13106                               \\ \hline
		1                          & 9000                                \\ \hline
	\end{tabular}
\end{table}

\begin{table}[h!]
	\centering
	\caption{Voltage at DACIN=32767}
	\begin{tabular}{ |c|c|c|c| }
		\hline
		\textbf{Signal} & \textbf{Channel} & \textbf{Min Voltage (V)} & \textbf{Max Voltage (V)} \\
		\hline
		vinn0           & 0                & 0.198                    & 5.1                      \\
		\hline
		vinp0           & 1                & 0.200                    & 5.16                     \\
		\hline
		vinn1           & 2                & 0.080                    & 5.1                      \\
		\hline
		vinp1           & 3                & 0.139                    & 5.12                     \\
		\hline
		vinn2           & 4                & 0.170                    & 5.11                     \\
		\hline
		vinp2           & 5                & 0.142                    & 5.08                     \\
		\hline
		vinn3           & 6                & 0.162                    & 5.09                     \\
		\hline
		vinp3           & 7                & 0.140                    & 5.07                     \\
		\hline
		vinn4           & 8                & 0.137                    & 5.06                     \\
		\hline
		vinp4           & 9                & 0.206                    & 5.1                      \\
		\hline
		vinn5           & 10               & 0.187                    & 5.08                     \\
		\hline
		vinp5           & 11               & 0.140                    & 5.07                     \\
		\hline
		vinn6           & 12               & 0.033                    & 5.01                     \\
		\hline
		vinp6           & 13               & 0.063                    & 5.02                     \\
		\hline
		vinn7           & 14               & 0.067                    & 5.01                     \\
		\hline
		vinp7           & 15               & 0.087                    & 5                        \\
		\hline
	\end{tabular}
\end{table}

\begin{table}[h!]
	\centering
	\caption{Voltage at DACIN=26214}
	\begin{tabular}{ |c|c|c|c| }
		\hline
		\textbf{Signal} & \textbf{Channel} & \textbf{Min Voltage (V)} & \textbf{Max Voltage (V)} \\
		\hline
		vinn0           & 0                & 0.181                    & 4.16                     \\
		\hline
		vinp0           & 1                & 0.185                    & 4.21                     \\
		\hline
		vinn1           & 2                & 0.066                    & 4.1                      \\
		\hline
		vinp1           & 3                & 0.120                    & 4.17                     \\
		\hline
		vinn2           & 4                & 0.160                    & 4.15                     \\
		\hline
		vinp2           & 5                & 0.130                    & 4.11                     \\
		\hline
		vinn3           & 6                & 0.153                    & 4.12                     \\
		\hline
		vinp3           & 7                & 0.133                    & 4.1                      \\
		\hline
		vinn4           & 8                & 0.120                    & 4.1                      \\
		\hline
		vinp4           & 9                & 0.196                    & 4.16                     \\
		\hline
		vinn5           & 10               & 0.181                    & 4.11                     \\
		\hline
		vinp5           & 11               & 0.130                    & 4.1                      \\
		\hline
		vinn6           & 12               & 0.040                    & 4.03                     \\
		\hline
		vinp6           & 13               & 0.060                    & 4.04                     \\
		\hline
		vinn7           & 14               & 0.060                    & 4.03                     \\
		\hline
		vinp7           & 15               & 0.070                    & 4.02                     \\
		\hline
	\end{tabular}
\end{table}

\begin{table}[h!]
	\centering
	\caption{Voltage at DACIN=19660}
	\begin{tabular}{ |c|c|c|c| }
		\hline
		\textbf{Signal} & \textbf{Channel} & \textbf{Min Voltage (V)} & \textbf{Max Voltage (V)} \\
		\hline
		vinn0           & 0                & 0.180                    & 3.16                     \\
		\hline
		vinp0           & 1                & 0.186                    & 3.19                     \\
		\hline
		vinn1           & 2                & 0.065                    & 3.11                     \\
		\hline
		vinp1           & 3                & 0.122                    & 3.15                     \\
		\hline
		vinn2           & 4                & 0.160                    & 3.12                     \\
		\hline
		vinp2           & 5                & 0.126                    & 3.11                     \\
		\hline
		vinn3           & 6                & 0.159                    & 3.14                     \\
		\hline
		vinp3           & 7                & 0.129                    & 3.11                     \\
		\hline
		vinn4           & 8                & 0.120                    & 3.11                     \\
		\hline
		vinp4           & 9                & 0.187                    & 3.15                     \\
		\hline
		vinn5           & 10               & 0.180                    & 3.14                     \\
		\hline
		vinp5           & 11               & 0.133                    & 3.11                     \\
		\hline
		vinn6           & 12               & 0.039                    & 3.04                     \\
		\hline
		vinp6           & 13               & 0.060                    & 3.06                     \\
		\hline
		vinn7           & 14               & 0.060                    & 3.05                     \\
		\hline
		vinp7           & 15               & 0.075                    & 3.04                     \\
		\hline
	\end{tabular}
\end{table}

\begin{table}[h!]
	\centering
	\caption{Voltage at DACIN=13106}
	\begin{tabular}{ |c|c|c|c| }
		\hline
		\textbf{Signal} & \textbf{Channel} & \textbf{Min Voltage (V)} & \textbf{Max Voltage (V)} \\
		\hline
		vinn0           & 0                & 0.184                    & 2.18                     \\
		\hline
		vinp0           & 1                & 0.188                    & 2.2                      \\
		\hline
		vinn1           & 2                & 0.065                    & 2.11                     \\
		\hline
		vinp1           & 3                & 0.122                    & 2.15                     \\
		\hline
		vinn2           & 4                & 0.159                    & 2.16                     \\
		\hline
		vinp2           & 5                & 0.126                    & 2.14                     \\
		\hline
		vinn3           & 6                & 0.157                    & 2.15                     \\
		\hline
		vinp3           & 7                & 0.128                    & 2.15                     \\
		\hline
		vinn4           & 8                & 0.120                    & 2.12                     \\
		\hline
		vinp4           & 9                & 0.191                    & 2.18                     \\
		\hline
		vinn5           & 10               & 0.180                    & 2.16                     \\
		\hline
		vinp5           & 11               & 0.135                    & 2.13                     \\
		\hline
		vinn6           & 12               & 0.039                    & 2.06                     \\
		\hline
		vinp6           & 13               & 0.060                    & 2.07                     \\
		\hline
		vinn7           & 14               & 0.060                    & 2.07                     \\
		\hline
		vinp7           & 15               & 0.066                    & 2.07                     \\
		\hline
	\end{tabular}
\end{table}

\begin{table}[h!]
	\centering
	\caption{Voltage at DACIN=9000}
	\begin{tabular}{ |c|c|c|c| }
		\hline
		\textbf{Signal} & \textbf{Channel} & \textbf{Min Voltage (V)} & \textbf{Max Voltage (V)} \\
		\hline
		vinn0           & 0                & 0.175                    & 1.53                     \\
		\hline
		vinp0           & 1                & 0.175                    & 1.54                     \\
		\hline
		vinn1           & 2                & 0.046                    & 1.44                     \\
		\hline
		vinp1           & 3                & 0.100                    & 1.49                     \\
		\hline
		vinn2           & 4                & 0.135                    & 1.5                      \\
		\hline
		vinp2           & 5                & 0.103                    & 1.48                     \\
		\hline
		vinn3           & 6                & 0.135                    & 1.5                      \\
		\hline
		vinp3           & 7                & 0.105                    & 1.48                     \\
		\hline
		vinn4           & 8                & 0.096                    & 1.47                     \\
		\hline
		vinp4           & 9                & 0.185                    & 1.53                     \\
		\hline
		vinn5           & 10               & 0.165                    & 1.51                     \\
		\hline
		vinp5           & 11               & 0.110                    & 1.47                     \\
		\hline
		vinn6           & 12               & 0.013                    & 1.4                      \\
		\hline
		vinp6           & 13               & 0.036                    & 1.41                     \\
		\hline
		vinn7           & 14               & 0.036                    & 1.41                     \\
		\hline
		vinp7           & 15               & 0.049                    & 1.41                     \\
		\hline
	\end{tabular}
\end{table}

